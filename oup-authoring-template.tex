%%
%% Copyright 2022 OXFORD UNIVERSITY PRESS
%%
%% This file is part of the 'oup-authoring-template Bundle'.
%% ---------------------------------------------
%%
%% It may be distributed under the conditions of the LaTeX Project Public
%% License, either version 1.2 of this license or (at your option) any
%% later version.  The latest version of this license is in
%%    http://www.latex-project.org/lppl.txt
%% and version 1.2 or later is part of all distributions of LaTeX
%% version 1999/12/01 or later.
%%
%% The list of all files belonging to the 'oup-authoring-template Bundle' is
%% given in the file `manifest.txt'.
%%
%% Template article for OXFORD UNIVERSITY PRESS's document class `oup-authoring-template'
%% with bibliographic references
%%

%%%CONTEMPORARY%%%
\documentclass[unnumsec,webpdf,contemporary,large]{oup-authoring-template}%
%\documentclass[unnumsec,webpdf,contemporary,large,namedate]{oup-authoring-template}% uncomment this line for author year citations and comment the above
%\documentclass[unnumsec,webpdf,contemporary,medium]{oup-authoring-template}
%\documentclass[unnumsec,webpdf,contemporary,small]{oup-authoring-template}

%%%MODERN%%%
%\documentclass[unnumsec,webpdf,modern,large]{oup-authoring-template}
%\documentclass[unnumsec,webpdf,modern,large,namedate]{oup-authoring-template}% uncomment this line for author year citations and comment the above
%\documentclass[unnumsec,webpdf,modern,medium]{oup-authoring-template}
%\documentclass[unnumsec,webpdf,modern,small]{oup-authoring-template}

%%%TRADITIONAL%%%
%\documentclass[unnumsec,webpdf,traditional,large]{oup-authoring-template}
%\documentclass[unnumsec,webpdf,traditional,large,namedate]{oup-authoring-template}% uncomment this line for author year citations and comment the above
%\documentclass[unnumsec,namedate,webpdf,traditional,medium]{oup-authoring-template}
%\documentclass[namedate,webpdf,traditional,small]{oup-authoring-template}

%\onecolumn % for one column layouts

%\usepackage{showframe}

\graphicspath{{Fig/}}

% line numbers
%\usepackage[mathlines, switch]{lineno}
%\usepackage[right]{lineno}

\theoremstyle{thmstyleone}%
\newtheorem{theorem}{Theorem}%  meant for continuous numbers
%%\newtheorem{theorem}{Theorem}[section]% meant for sectionwise numbers
%% optional argument [theorem] produces theorem numbering sequence instead of independent numbers for Proposition
\newtheorem{proposition}[theorem]{Proposition}%
%%\newtheorem{proposition}{Proposition}% to get separate numbers for theorem and proposition etc.
\theoremstyle{thmstyletwo}%
\newtheorem{example}{Example}%
\newtheorem{remark}{Remark}%
\theoremstyle{thmstylethree}%
\newtheorem{definition}{Definition}

\begin{document}

\journaltitle{Journal Title Here}
\DOI{DOI HERE}
\copyrightyear{2022}
\pubyear{2019}
\access{Advance Access Publication Date: Day Month Year}
\appnotes{Paper}

\firstpage{1}

%\subtitle{Subject Section}

\title[nf-core/viralgenie: A Novel Pipeline For Untargeted Viral Genome Reconstruction]{nf-core/viralgenie: A Novel Pipeline For Untargeted Viral Genome Reconstruction}

\author[1,$\ast$]{Joon Klaps\ORCID{0000-0002-2507-0430}}
\author[1]{Philippe Lemey\ORCID{0000-0003-2826-5353}}
\author[1]{Liana Kafetzopoulou\ORCID{0000-0003-4531-1374}}

\authormark{Joon Klaps et al.}

\address[1]{\orgdiv{Rega Institute for Medical Research Department of Microbiology, Immunology and Transplantation Department of Pharmaceutical and Pharmacological Sciences
}, \orgname{KU Leuven}, \orgaddress{\street{Herestraat 49}, \postcode{3000}, \state{Leuven}, \country{Belgium}}}
% \address[2]{\orgdiv{Department}, \orgname{Organization}, \orgaddress{\street{Street}, \postcode{Postcode}, \state{State}, \country{Country}}}
% \address[3]{\orgdiv{Department}, \orgname{Organization}, \orgaddress{\street{Street}, \postcode{Postcode}, \state{State}, \country{Country}}}
% \address[4]{\orgdiv{Department}, \orgname{Organization}, \orgaddress{\street{Street}, \postcode{Postcode}, \state{State}, \country{Country}}}

\corresp[$\ast$]{Corresponding author. \href{email:joon.klaps@kuleuven.be}{joon.klaps@kuleuven.be}}

\received{Date}{0}{Year}
\revised{Date}{0}{Year}
\accepted{Date}{0}{Year}

%\editor{Associate Editor: Name}

\abstract{
\textbf{Motivation:} Eukaryotic viruses present significant challenges in genome reconstruction and variant analysis due to their extensive diversity, quasi-species, absence of universal marker genes, and genome segmentation. While de novo assembly followed by reference database matching and consequently, read mapping is a common approach, manual execution of this workflow is extremely time-consuming, particularly due to the extensive reference verification and selection required. There is a critical need for an automated, scalable pipeline that can efficiently handle viral metagenomic analysis without manual intervention.\\
\textbf{Results:} Here, we present nf-core/viralgenie, a comprehensive viral metagenomic pipeline for untargeted genome reconstruction, and variant analysis of eukaryotic viruses. Viralgenie is implemented as a modular Nextflow workflow that processes metagenomic and hybridization capture enriched samples to automatically detect and assemble viral genomes, while also performing variant analysis. The pipeline features automated reference selection, quality control metrics, comprehensive documentation, and seamless integration with containerization technologies including Docker, Singularity, and Podman. We demonstrate its utility and accuracy through validation on both simulated and real datasets, showing robust performance across diverse viral families and sample types.\\
\textbf{Availability:} nf-core/viralgenie is freely available at https://github.com/nf-core/viralgenie with comprehensive documentation at https://nf-co.re/viralgenie.\\
% \textbf{Contact:} \href{joon.klaps@kuleuven.be}{joon.klaps@kuleuven.be}\\
% \textbf{Supplementary information:} Supplementary data are available at \textit{Bioinformatics} online.
}
\keywords{viralgenie, bioinformatic pipeline, nextflow, viral metagenomics, viral assembly, viral variant analysis}

% \boxedtext{
% \begin{itemize}
% \item Key boxed text here.
% \item Key boxed text here.
% \item Key boxed text here.
% \end{itemize}}

\maketitle


\section{Introduction}\label{sec1}

Reconstructing viral genomes from metagenomic sequencing data presents significant computational challenges, particularly for eukaryotic viruses that exhibit extensive genetic diversity, and quasi-species formation. This diversity is further compounded by the prevalence of segmented genomes in many viral families, including influenza, rotavirus, and bunyaviruses, where individual segments may undergo independent evolutionary pressures and reassortment events.

Resulting in a challenging landscape for viral genome reconstruction. Typically, for accurate and complete viral genome reconstruction, manual curation of contigs and reference matching is required. This process is not only time-consuming making it impractical for large-scale studies or rapid response scenarios such as emerging viral outbreaks of unknown origin. The need for a more automated and scalable solution has become increasingly apparent, particularly in the context of public health surveillance and epidemiological research [NEED FOR CITATION].

% Current bioinformatics solutions in the nf-core ecosystem address specific aspects of viral analysis but lack comprehensive coverage for untargeted viral genome reconstruction. The nf-core/viralrecon pipeline excels in targeted analysis scenarios where reference genomes are predetermined, making it well-suited for surveillance of known pathogens such as SARS-CoV-2. However, this targeted approach becomes limiting when analyzing environmental samples, investigating novel viral outbreaks, or studying highly divergent viral strains where the appropriate reference genome is unknown a priori. Similarly, nf-core/mag focuses on comprehensive metagenomic analysis but is optimized for describing entire microbial communities rather than specifically targeting viral diversity.

% The gap between targeted and metagenomic approaches becomes particularly pronounced when studying intrahost viral evolution, viral co-infections, or reassortment events in segmented viruses. These research questions require the detection and reconstruction of multiple related viral genomes from the same sample, a capability that existing pipelines struggle to provide efficiently. Furthermore, the lack of standardized workflows for untargeted viral analysis hinders reproducibility and limits the adoption of best practices across the viral genomics community.

To address these limitations, we developed nf-core/viralgenie, a comprehensive pipeline specifically designed for untargeted viral genome reconstruction. The pipeline implements an automated workflow that performs de novo assembly, reference matching through sequence clustering, and iterative refinement by read mapping and consensus calling to reconstruct viral genomes without prior knowledge of the target sequences. By integrating containerization technologies and following nf-core standards, viralgenie ensures reproducibility and scalability across diverse computational environments while maintaining the flexibility required for varied research applications.

\section{Methods}\label{sec2}

The nf-core/viralgenie pipeline implements a comprehensive workflow for untargeted viral genome reconstruction and variant analysis, consisting of five major analytical modules: preprocessing, metagenomic diversity assessment, assembly and polishing, variant analysis with iterative refinement, and consensus quality control. The pipeline is implemented in Nextflow and follows nf-core standards, ensuring reproducibility and portability across computational environments through containerization with Docker, Singularity, or Conda.

\subsection{Pipeline Overview and Installation}\label{subsec_overview}

Viralgenie requires Nextflow and a container management system (Docker, Singularity, or Conda). The pipeline can be executed with minimal setup:

\begin{verbatim}
nextflow run nf-core/viralgenie \
    -profile docker \
    --input samplesheet.csv
\end{verbatim}

Input data is provided through a samplesheet in CSV, TSV, YAML, or JSON format containing sample names and paths to FASTQ files. The pipeline supports both single-end and paired-end sequencing data, with optional support for Unique Molecular Identifiers (UMIs) and mapping constraints for reference-guided analysis.

\subsection{Read Preprocessing}\label{subsec_preprocessing}

The preprocessing module performs quality control and filtering of raw sequencing reads through five sequential steps. Initial quality assessment is conducted using FastQC before and after each processing step to monitor data quality throughout the workflow.

Adapter trimming and read processing is performed using either fastp (default) or Trimmomatic, both of which provide comprehensive adapter removal and quality filtering capabilities. For libraries prepared with UMIs, PCR duplicate removal is implemented using HUMID, which supports both directional and maximum clustering methods for UMI-based deduplication. The directional method (default) accounts for expected PCR errors by grouping reads using the relationship: node A counts $\geq$ (2 $\times$ node B counts) - 1.

Read merging is performed when multiple sequencing runs exist for the same sample, concatenating R1 files with R1 and R2 files with R2, while maintaining separation between single-end and paired-end data. Complexity filtering, implemented through BBduk or prinseq++, removes low-complexity sequences containing repetitive elements that could produce spurious alignments during downstream analysis.

Host contamination removal is performed using Kraken2 against a user-specified host genome database. The default database contains a subset of the human genome, though users are strongly encouraged to employ more comprehensive databases including complete host genomes, common sequencer contaminants, and bacterial genomes to ensure thorough decontamination.

\subsection{Metagenomic Diversity Assessment}\label{subsec_diversity}

Taxonomic classification of processed reads is performed using two complementary approaches to maximize detection sensitivity across diverse viral families. Kaiju performs protein-based classification using a Burrows-Wheeler transform search strategy against annotated protein-coding genes from microbial genomes, enabling detection of highly divergent sequences through amino acid conservation. Kraken2 provides DNA-level classification using k-mer mapping to identify the lowest common ancestor (LCA) of genomes containing specific k-mers. Optional Bracken analysis can be enabled for abundance estimation, though viral abundance comparisons should be interpreted cautiously due to the absence of universal marker genes in viruses.

Results from both classifiers are visualized using Krona, which generates interactive multi-layered pie charts allowing hierarchical exploration of taxonomic diversity. This dual-classification approach compensates for the limitations of individual methods and provides comprehensive coverage of viral diversity in metagenomic samples.

\subsection{Assembly and Polishing}\label{subsec_assembly}

The assembly module implements a multi-assembler approach followed by sophisticated clustering and scaffolding procedures. De novo assembly is performed using three complementary assemblers: SPAdes (configured for RNA viral mode by default), MEGAHIT, and Trinity. This multi-assembler strategy capitalizes on the distinct algorithmic strengths of each tool to maximize genome recovery across diverse viral families and coverage distributions.

Assembled contigs undergo extension using SSPACE Basic, which leverages paired-end read information to scaffold and extend initial assemblies. Coverage calculation is performed by mapping processed reads back to contigs using BWAmem2 (default), BWA, or Bowtie2, enabling identification and filtration of low-coverage assemblies that likely represent assembly artifacts.

Reference matching is conducted through BLASTn searches against a comprehensive reference sequence pool, with the default being the latest clustered Reference Viral Database (RVDB). The top five BLAST hits for each contig are retained and incorporated into subsequent clustering analysis, facilitating identification of related genomic segments and appropriate reference sequences for scaffolding.

Taxonomy-guided clustering employs a two-stage process to group related contigs. Initial pre-clustering uses taxonomic assignments from both Kraken2 and Kaiju to resolve classification inconsistencies and separate contigs by taxonomic identity. Subsequent nucleotide similarity clustering is performed using one of six available algorithms: CD-HIT-EST, VSEARCH, MMseqs-linclust, MMseqs-cluster, vRhyme, or Mash with network-based community detection. The choice of clustering method allows optimization for specific dataset characteristics, with CD-HIT-EST providing speed for smaller datasets and MMseqs variants offering scalability for larger analyses.

Final scaffolding maps all cluster members to their respective centroids using Minimap2, followed by consensus calling with iVar to generate reference-assisted assemblies. Regions with zero coverage depth are optionally annotated using reference sequences to produce complete genome reconstructions.

\subsection{Variant Analysis and Iterative Refinement}\label{subsec_variant}

The variant calling module supports two distinct analytical pathways: external reference-based analysis and de novo assembly refinement. In external reference-based analysis, users provide reference genomes through mapping constraints, with automatic selection of the most appropriate references using Mash k-mer distance calculations. This approach selects references sharing the highest number of k-mers with the sequencing reads, minimizing mapping bias for highly divergent viral sequences.

For de novo assembly refinement, the pipeline performs iterative improvement of initially assembled consensus genomes. Each iteration maps reads back to the current consensus using BWAmem2, BWA, or Bowtie2, followed by variant calling and consensus generation. The default configuration performs two refinement iterations, though this is user-configurable.

Variant calling is implemented using either BCFtools or iVar, each offering distinct advantages for viral genomics applications. BCFtools provides higher precision through sophisticated statistical modeling but may miss low-frequency variants. iVar excels at detecting multiallelic sites and low-frequency variants, making it particularly suitable for viral quasi-species analysis. iVar also handles ambiguous nucleotides more effectively, representing multiallelic positions with IUPAC ambiguity codes rather than masking them.

Optional UMI-based deduplication can be performed using UMI-tools, while standard PCR duplicate removal utilizes Picard MarkDuplicates. Comprehensive mapping statistics are generated using samtools (flagstat, idxstats, stats), Picard CollectMultipleMetrics, and mosdepth for coverage analysis.

Variant filtering removes variants with insufficient depth or quality, with BCFtools implementing additional steps to handle multiallelic sites and merge SNPs with indels. The final consensus sequences incorporate high-quality variants while maintaining genomic completeness through reference-guided gap filling.

\subsection{Consensus Quality Control and Annotation}\label{subsec_qc}

Comprehensive quality assessment of reconstructed viral genomes is performed through multiple complementary analyses. QUAST provides standard assembly metrics including contig statistics, N50 values, and quantification of ambiguous bases, which serves as a primary indicator of consensus quality. CheckV estimates genome completeness and contamination by comparison against a curated database of complete viral genomes, though completeness estimates for segmented viruses should be interpreted considering that CheckV calculates completeness based on concatenated segment lengths.

Functional annotation is performed using Prokka, which identifies coding sequences and assigns functional annotations. While originally designed for bacterial genomes, Prokka provides reasonable annotation for viral sequences, particularly when supplemented with custom viral protein databases such as prot-RVDB.

Consensus genomes undergo similarity analysis through BLASTn searches against the reference pool and MMseqs searches against comprehensive annotation databases such as Virosaurus. MMseqs enables rapid tblastx-equivalent searches for highly divergent sequences while maintaining nucleotide database compatibility. The annotation pipeline extracts species identification, segment designation, expected host information, and additional metadata from the best database matches.

Multiple sequence alignment using MAFFT aligns final consensus genomes with their corresponding references and constituent de novo contigs, enabling assessment of assembly accuracy and identification of genomic variations. Variant functional annotation is performed using SnpEff, which predicts the biological impact of detected variants, including synonymous/non-synonymous classifications and amino acid changes.

All quality control metrics are integrated into interactive MultiQC reports, providing comprehensive visualization of pipeline results. Custom summary tables extract key metrics from each analysis tool, facilitating rapid assessment of reconstruction quality across multiple samples.

\section{Results}\label{sec3}

The nf-core/viralgenie pipeline provides comprehensive outputs enabling thorough evaluation of viral genome reconstruction quality and downstream analysis preparation. This section describes the expected outputs, runtime characteristics, and parameter selection rationale that guide optimal pipeline performance.

\subsection{Pipeline Performance and Runtime Metrics}\label{subsec_performance}

Viralgenie demonstrates efficient computational performance across diverse sample types and scales. Runtime scales primarily with read depth and viral diversity, with typical processing times ranging from 2-6 hours for standard metagenomic samples (10-50 million reads) on modern compute clusters. The multi-assembler approach adds computational overhead compared to single-assembler pipelines but provides superior genome recovery, particularly for highly divergent or low-coverage viral sequences.

Memory requirements vary by analysis module, with assembly typically representing the most resource-intensive step. SPAdes requires the highest memory allocation (8-32 GB depending on dataset size), while MEGAHIT and Trinity offer more memory-efficient alternatives. The clustering and variant calling steps scale efficiently with read depth, maintaining reasonable resource requirements even for high-coverage datasets.

Pipeline scalability benefits from Nextflow's built-in parallelization capabilities, enabling concurrent processing of multiple samples and assembly methods. Resource allocation can be customized through configuration profiles, allowing optimization for different computational environments from local workstations to high-performance computing clusters.

\subsection{Output Organization and Interpretation}\label{subsec_output}

Viralgenie generates a hierarchical output structure designed for intuitive navigation and comprehensive result interpretation. The primary MultiQC report serves as the central hub for quality assessment, presenting interactive visualizations of all major pipeline metrics. Custom summary tables within the MultiQC report extract key information from each analysis tool, enabling rapid identification of high-quality consensus genomes and potential issues requiring attention.

Consensus sequences are organized by sample and clustering results, with clear naming conventions indicating assembly methods and refinement iterations. Each consensus genome is accompanied by comprehensive metadata including quality metrics, annotation results, and mapping statistics. Intermediate files are preserved to enable detailed troubleshooting and alternative parameter exploration.

The \texttt{overview-tables} directory contains summarized results from all major analysis steps, providing convenient access to quantitative metrics for downstream analysis or publication. These tables include assembly statistics, taxonomy assignments, variant calling results, and quality control metrics in standardized formats compatible with common statistical software packages.

\subsection{Default Parameter Selection and Tool Choices}\label{subsec_parameters}

The pipeline's default parameters reflect extensive benchmarking and optimization for viral metagenomic applications. Tool selection balances computational efficiency with analytical sensitivity, prioritizing methods that perform well across diverse viral families and sample types.

For clustering applications, the default CD-HIT-EST algorithm with 85\% similarity threshold provides an optimal balance between sensitivity and specificity for most viral datasets. This threshold effectively groups related genomic segments while maintaining separation of distinct viral strains. Alternative clustering methods are provided to accommodate specific research needs: VSEARCH for enhanced accuracy, MMseqs variants for scalability, and Mash for rapid approximate clustering.

Variant calling defaults favor iVar over BCFtools for consensus generation due to its superior handling of viral-specific challenges including multiallelic sites, low-frequency variants, and ambiguous base calling. However, BCFtools is employed for intermediate refinement steps where its conservative approach helps prevent error propagation during iterative improvement.

Database selections prioritize comprehensive coverage while maintaining computational tractability. The clustered RVDB serves as the default reference pool, providing broad viral representation while limiting computational requirements. For taxonomic classification, viral-specific databases are employed to maximize detection sensitivity for eukaryotic viruses while minimizing false positive assignments from bacterial or archaeal sequences.

Quality thresholds are conservatively set to ensure high-confidence results while accommodating the inherent challenges of viral genome reconstruction. Minimum read depth requirements (10x for variant calling), quality scores (Phred 20), and coverage thresholds are calibrated based on empirical performance across diverse viral families and sample preparation methods.

\section{Discussion}\label{sec4}

Lorem ipsum dolor sit amet, consectetur adipiscing elit, sed do eiusmod tempor incididunt ut labore et dolore magna aliqua. Ut enim ad minim veniam, quis nostrud exercitation ullamco laboris nisi ut aliquip ex ea commodo consequat. Duis aute irure dolor in reprehenderit in voluptate velit esse cillum dolore eu fugiat nulla pariatur. Excepteur sint occaecat cupidatat non proident, sunt in culpa qui officia deserunt mollit anim id est laborum.

\paragraph{This is an example for fourth level head - paragraph head}

Lorem ipsum dolor sit amet, consectetur adipiscing elit, sed do eiusmod tempor incididunt ut labore et dolore magna aliqua. Ut enim ad minim veniam, quis nostrud exercitation ullamco laboris nisi ut aliquip ex ea commodo consequat. Duis aute irure dolor in reprehenderit in voluptate velit esse cillum dolore eu fugiat nulla pariatur. Excepteur sint occaecat cupidatat non proident, sunt in culpa qui officia deserunt mollit anim id est laborum.


\section{Equations}\label{sec4}

Equations in \LaTeX{} can either be inline or set as display equations. For
inline equations use the \verb+$...$+ commands. Eg: the equation
$H\psi = E \psi$ is written via the command \verb+$H \psi = E \psi$+.

For display equations (with auto generated equation numbers)
one can use the equation or eqnarray environments:
\begin{equation}
\|\tilde{X}(k)\|^2 \leq\frac{\sum\limits_{i=1}^{p}\left\|\tilde{Y}_i(k)\right\|^2+\sum\limits_{j=1}^{q}\left\|\tilde{Z}_j(k)\right\|^2 }{p+q},\label{eq1}
\end{equation}
where,
\begin{align}
D_\mu &=  \partial_\mu - ig \frac{\lambda^a}{2} A^a_\mu \nonumber \\
F^a_{\mu\nu} &= \partial_\mu A^a_\nu - \partial_\nu A^a_\mu + g f^{abc} A^b_\mu A^a_\nu.\label{eq2}
\end{align}
Notice the use of \verb+\nonumber+ in the align environment at the end
of each line, except the last, so as not to produce equation numbers on
lines where no equation numbers are required. The \verb+\label{}+ command
should only be used at the last line of an align environment where
\verb+\nonumber+ is not used.
\begin{equation}
Y_\infty = \left( \frac{m}{\textrm{GeV}} \right)^{-3}
    \left[ 1 + \frac{3 \ln(m/\textrm{GeV})}{15}
    + \frac{\ln(c_2/5)}{15} \right].
\end{equation}
The class file also supports the use of \verb+\mathbb{}+, \verb+\mathscr{}+ and
\verb+\mathcal{}+ commands. As such \verb+\mathbb{R}+, \verb+\mathscr{R}+
and \verb+\mathcal{R}+ produces $\mathbb{R}$, $\mathscr{R}$ and $\mathcal{R}$
respectively (refer Subsubsection~\ref{subsubsec3}).


Lorem ipsum dolor sit amet, consectetur adipiscing elit, sed do
eiusmod tempor incididunt ut labore et dolore magna aliqua. Ut enim ad minim veniam, quis nostrud exercitation ullamco laboris nisi ut aliquip ex ea commodo consequat. Duis aute irure dolor in reprehenderit in voluptate velit esse cillum dolore eu fugiat nulla pariatur. Excepteur sint occaecat cupidatat non proident, sunt in culpa qui officia deserunt mollit anim id est laborum. Lorem ipsum dolor sit amet, consectetur adipiscing elit, sed do
eiusmod tempor incididunt ut labore et dolore magna aliqua. Ut enim ad minim veniam, quis nostrud exercitation ullamco laboris nisi ut aliquip ex ea commodo consequat. Duis aute irure dolor in reprehenderit in voluptate velit esse cillum dolore eu fugiat nulla pariatur. Excepteur sint occaecat cupidatat non proident, sunt in culpa qui officia deserunt mollit anim id est laborum.
Lorem ipsum dolor sit amet, consectetur adipiscing elit, sed do
eiusmod tempor incididunt ut labore et dolore magna aliqua. Ut enim ad minim veniam, quis nostrud exercitation ullamco laboris nisi ut aliquip ex ea commodo consequat.


\section{Tables}\label{sec5}

Tables can be inserted via the normal table and tabular environment. To put
footnotes inside tables one has to Lorem ipsum dolor sit amet, consectetur adipiscing elit, sed do eiusmod tempor incididunt ut labore et dolore magna aliqua. Ut enim ad minim veniam, quis nostrud exercitation ullamco laboris nisi ut aliquip ex ea commodo consequat. Duis aute irure dolor in reprehenderit in voluptate velit esse cillum dolore eu fugiat nulla pariatur. Excepteur sint occaecat cupidatat non proident, sunt in culpa qui officia deserunt mollit anim id est laborum. use the additional ``tablenotes" environment
enclosing the tabular environment. The footnote appears just below the table
itself (refer Tables~\ref{tab1} and \ref{tab2}).


\begin{verbatim}
\begin{table}[t]
\begin{center}
\begin{minipage}{<width>}
\caption{<table-caption>\label{<table-label>}}%
\begin{tabular}{@{}llll@{}}
\toprule
column 1 & column 2 & column 3 & column 4\\
\midrule
row 1 & data 1 & data 2          & data 3 \\
row 2 & data 4 & data 5$^{1}$ & data 6 \\
row 3 & data 7 & data 8      & data 9$^{2}$\\
\botrule
\end{tabular}
\begin{tablenotes}%
\item Source: Example for source.
\item[$^{1}$] Example for a 1st table footnote.
\item[$^{2}$] Example for a 2nd table footnote.
\end{tablenotes}
\end{minipage}
\end{center}
\end{table}
\end{verbatim}


Lengthy tables which do not fit within textwidth should be set as rotated tables. For this, we need to use \verb+\begin{sidewaystable}...+ \verb+\end{sidewaystable}+ instead of\break \verb+\begin{table}...+ \verb+\end{table}+ environment.


\begin{table}[!t]
\caption{Caption text\label{tab1}}%
\begin{tabular*}{\columnwidth}{@{\extracolsep\fill}llll@{\extracolsep\fill}}
\toprule
column 1 & column 2  & column 3 & column 4\\
\midrule
row 1    & data 1   & data 2  & data 3  \\
row 2    & data 4   & data 5$^{1}$  & data 6  \\
row 3    & data 7   & data 8  & data 9$^{2}$  \\
\botrule
\end{tabular*}
\begin{tablenotes}%
\item Source: This is an example of table footnote this is an example of table footnote this is an example of table footnote this is an example of~table footnote this is an example of table footnote
\item[$^{1}$] Example for a first table footnote.
\item[$^{2}$] Example for a second table footnote.
\end{tablenotes}
\end{table}

\begin{table*}[t]
\caption{Example of a lengthy table which is set to full textwidth.\label{tab2}}
\tabcolsep=0pt%%
\begin{tabular*}{\textwidth}{@{\extracolsep{\fill}}lcccccc@{\extracolsep{\fill}}}
\toprule%
& \multicolumn{3}{@{}c@{}}{Element 1$^{1}$} & \multicolumn{3}{@{}c@{}}{Element 2$^{2}$} \\
\cline{2-4}\cline{5-7}%
Project & Energy & $\sigma_{calc}$ & $\sigma_{expt}$ & Energy & $\sigma_{calc}$ & $\sigma_{expt}$ \\
\midrule
Element 3  & 990 A & 1168 & $1547\pm12$ & 780 A & 1166 & $1239\pm100$\\
Element 4  & 500 A & 961  & $922\pm10$  & 900 A & 1268 & $1092\pm40$\\
\botrule
\end{tabular*}
\begin{tablenotes}%
\item Note: This is an example of table footnote this is an example of table footnote this is an example of table footnote this is an example of~table footnote this is an example of table footnote
\item[$^{1}$] Example for a first table footnote.
\item[$^{2}$] Example for a second table footnote.\vspace*{6pt}
\end{tablenotes}
\end{table*}

\begin{sidewaystable}%[!p]
\caption{Tables which are too long to fit, should be written using the ``sidewaystable" environment as shown here\label{tab3}}
\tabcolsep=0pt%
\begin{tabular*}{\textwidth}{@{\extracolsep{\fill}}lcccccc@{\extracolsep{\fill}}}
\toprule%
& \multicolumn{3}{@{}c@{}}{Element 1$^{1}$}& \multicolumn{3}{@{}c@{}}{Element$^{2}$} \\
\cline{2-4}\cline{5-7}%
Projectile & Energy     & $\sigma_{calc}$ & $\sigma_{expt}$ & Energy & $\sigma_{calc}$ & $\sigma_{expt}$ \\
\midrule
Element 3 & 990 A & 1168 & $1547\pm12$ & 780 A & 1166 & $1239\pm100$ \\
Element 4 & 500 A & 961  & $922\pm10$  & 900 A & 1268 & $1092\pm40$ \\
\botrule
\end{tabular*}
\begin{tablenotes}%
\item Note: This is an example of a table footnote this is an example of a table footnote this is an example of a table footnote this is an example of a table footnote this is an example of a table footnote
\item[$^{1}$] This is an example of a table footnote
\end{tablenotes}
\end{sidewaystable}


\section{Figures}\label{sec6}

As per display \LaTeX\ standards one has to use eps images for \verb+latex+ compilation and \verb+pdf/jpg/png+ images for
\verb+pdflatex+ compilation. This is one of the major differences between \verb+latex+
and \verb+pdflatex+. The images should be single-page documents. The command for inserting images
for \verb+latex+ and \verb+pdflatex+ can be generalized. The package used to insert images in \verb+latex/pdflatex+ is the
graphicx package. Figures can be inserted via the normal figure environment as shown in the below example:


\begin{figure}[!t]%
\centering
{\color{black!20}\rule{213pt}{37pt}}
\caption{This is a widefig. This is an example of a long caption this is an example of a long caption  this is an example of a long caption this is an example of a long caption}\label{fig1}
\end{figure}

\begin{figure*}[!t]%
\centering
{\color{black!20}\rule{438pt}{74pt}}
\caption{This is a widefig. This is an example of a long caption this is an example of a long caption  this is an example of a long caption this is an example of a long caption}\label{fig2}
\end{figure*}


\begin{verbatim}
\begin{figure}[t]
        \centering\includegraphics{<eps-file>}
        \caption{<figure-caption>}
        \label{<figure-label>}
\end{figure}
\end{verbatim}

Test text here.

For sample purposes, we have included the width of images in the
optional argument of \verb+\includegraphics+ tag. Please ignore this.
Lengthy figures which do not fit within textwidth should be set in rotated mode. For rotated figures, we need to use \verb+\begin{sidewaysfigure}+ \verb+...+ \verb+\end{sidewaysfigure}+ instead of the \verb+\begin{figure}+ \verb+...+ \verb+\end{figure}+ environment.

\begin{sidewaysfigure}%
\centering
{\color{black!20}\rule{610pt}{102pt}}
\caption{This is an example for a sideways figure. This is an example of a long caption this is an example of a long caption  this is an example of a long caption this is an example of a long caption}\label{fig3}
\end{sidewaysfigure}



\section{Algorithms, Program codes and Listings}\label{sec7}

Packages \verb+algorithm+, \verb+algorithmicx+ and \verb+algpseudocode+ are used for setting algorithms in latex.
For this, one has to use the below format:


\begin{verbatim}
\begin{algorithm}
\caption{<alg-caption>}\label{<alg-label>}
\begin{algorithmic}[1]
. . .
\end{algorithmic}
\end{algorithm}
\end{verbatim}


You may need to refer to the above-listed package documentations for more details before setting an \verb+algorithm+ environment.
To set program codes, one has to use the \verb+program+ package. We need to use the \verb+\begin{program}+ \verb+...+
\verb+\end{program}+ environment to set program codes.

\begin{algorithm}[!t]
\caption{Calculate $y = x^n$}\label{algo1}
\begin{algorithmic}[1]
\Require $n \geq 0 \vee x \neq 0$
\Ensure $y = x^n$
\State $y \Leftarrow 1$
\If{$n < 0$}
        \State $X \Leftarrow 1 / x$
        \State $N \Leftarrow -n$
\Else
        \State $X \Leftarrow x$
        \State $N \Leftarrow n$
\EndIf
\While{$N \neq 0$}
        \If{$N$ is even}
            \State $X \Leftarrow X \times X$
            \State $N \Leftarrow N / 2$
        \Else[$N$ is odd]
            \State $y \Leftarrow y \times X$
            \State $N \Leftarrow N - 1$
        \EndIf
\EndWhile
\end{algorithmic}
\end{algorithm}

Similarly, for \verb+listings+, one has to use the \verb+listings+ package. The \verb+\begin{lstlisting}+ \verb+...+ \verb+\end{lstlisting}+ environment is used to set environments similar to the \verb+verbatim+ environment. Refer to the \verb+lstlisting+ package documentation for more details on this.


\begin{minipage}{\hsize}%
\lstset{language=Pascal}% Set your language (you can change the language for each code-block optionally)
\begin{lstlisting}[frame=single,framexleftmargin=-1pt,framexrightmargin=-17pt,framesep=12pt,linewidth=0.98\textwidth]
for i:=maxint to 0 do
begin
{ do nothing }
end;
Write('Case insensitive ');
Write('Pascal keywords.');
\end{lstlisting}
\end{minipage}


\section{Cross referencing}\label{sec8}

Environments such as figure, table, equation, and align can have a label
declared via the \verb+\label{#label}+ command. For figures and table
environments one should use the \verb+\label{}+ command inside or just
below the \verb+\caption{}+ command.  One can then use the
\verb+\ref{#label}+ command to cross-reference them. As an example, consider
the label declared for Figure \ref{fig1} which is
\verb+\label{fig1}+. To cross-reference it, use the command
\verb+ Figure \ref{fig1}+, for which it comes up as
``Figure~\ref{fig1}".

\subsection{Details on reference citations}\label{subsec3}

With standard numerical .bst files, only numerical citations are possible.
With an author-year .bst file, both numerical and author-year citations are possible.

If author-year citations are selected, \verb+\bibitem+ must have one of the following forms:


{\footnotesize%
\begin{verbatim}
\bibitem[Jones et al.(1990)]{key}...
\bibitem[Jones et al.(1990)Jones,
                Baker, and Williams]{key}...
\bibitem[Jones et al., 1990]{key}...
\bibitem[\protect\citeauthoryear{Jones,
                Baker, and Williams}
                {Jones et al.}{1990}]{key}...
\bibitem[\protect\citeauthoryear{Jones et al.}
                {1990}]{key}...
\bibitem[\protect\astroncite{Jones et al.}
                {1990}]{key}...
\bibitem[\protect\citename{Jones et al., }
                1990]{key}...
\harvarditem[Jones et al.]{Jones, Baker, and
                Williams}{1990}{key}...
\end{verbatim}}


This is either to be made up manually, or to be generated by an
appropriate .bst file with BibTeX. Then,


{%
\begin{verbatim}
                    Author-year mode
                        || Numerical mode
\citet{key} ==>>  Jones et al. (1990)
                        || Jones et al. [21]
\citep{key} ==>> (Jones et al., 1990) || [21]
\end{verbatim}}


\noindent
Multiple citations as normal:


{%
\begin{verbatim}
\citep{key1,key2} ==> (Jones et al., 1990;
                         Smith, 1989)||[21,24]
        or (Jones et al., 1990, 1991)||[21,24]
        or (Jones et al., 1990a,b)   ||[21,24]
\end{verbatim}}


\noindent
\verb+\cite{key}+ is the equivalent of \verb+\citet{key}+ in author-year mode
and  of \verb+\citep{key}+ in numerical mode. Full author lists may be forced with
\verb+\citet*+ or \verb+\citep*+, e.g.


{%
\begin{verbatim}
\citep*{key} ==>> (Jones, Baker, and Mark, 1990)
\end{verbatim}}


\noindent
Optional notes as:


{%
\begin{verbatim}
\citep[chap. 2]{key}     ==>>
        (Jones et al., 1990, chap. 2)
\citep[e.g.,][]{key}     ==>>
        (e.g., Jones et al., 1990)
\citep[see][pg. 34]{key} ==>>
        (see Jones et al., 1990, pg. 34)
\end{verbatim}}


\noindent
(Note: in standard LaTeX, only one note is allowed, after the ref.
Here, one note is like the standard, two make pre- and post-notes.)


{%
\begin{verbatim}
\citealt{key}   ==>> Jones et al. 1990
\citealt*{key}  ==>> Jones, Baker, and
                        Williams 1990
\citealp{key}   ==>> Jones et al., 1990
\citealp*{key}  ==>> Jones, Baker, and
                        Williams, 1990
\end{verbatim}}


\noindent
Additional citation possibilities (both author-year and numerical modes):


{%
\begin{verbatim}
\citeauthor{key}       ==>> Jones et al.
\citeauthor*{key}      ==>> Jones, Baker, and
                                Williams
\citeyear{key}         ==>> 1990
\citeyearpar{key}      ==>> (1990)
\citetext{priv. comm.} ==>> (priv. comm.)
\citenum{key}          ==>> 11 [non-superscripted]
\end{verbatim}}


\noindent
Note: full author lists depend on whether the bib style supports them;
if not, the abbreviated list is printed even when full is requested.

\noindent
For names like della Robbia at the start of a sentence, use


{%
\begin{verbatim}
\Citet{dRob98}      ==>> Della Robbia (1998)
\Citep{dRob98}      ==>> (Della Robbia, 1998)
\Citeauthor{dRob98} ==>> Della Robbia
\end{verbatim}}


\noindent
The following is an example for \verb+\cite{...}+: \cite{rahman2019centroidb}. Another example for \verb+\citep{...}+: \citep{bahdanau2014neural,imboden2018cardiorespiratory,motiian2017unified,murphy2012machine,ji20123d}.
Sample cites here \cite{krizhevsky2012imagenet,horvath2018dna} and \cite{pyrkov2018quantitative}, \cite{wang2018face}, \cite{lecun2015deep,zhang2018fine,ravi2016deep}.


\section{Lists}\label{sec9}

List in \LaTeX{} can be of three types: numbered, bulleted and unnumbered. The ``enumerate'' environment produces a numbered list, the
``itemize'' environment produces a bulleted list and the ``unlist''
environment produces an unnumbered list.
In each environment, a new entry is added via the \verb+\item+ command.
\begin{enumerate}[1.]
\item This is the 1st item

\item Enumerate creates numbered lists, itemize creates bulleted lists and
unnumerate creates unnumbered lists.
\begin{enumerate}[(a)]
\item Second level numbered list. Enumerate creates numbered lists, itemize creates bulleted lists and
description creates unnumbered lists.

\item Second level numbered list. Enumerate creates numbered lists, itemize creates bulleted lists and
description creates unnumbered lists.
\begin{enumerate}[(ii)]
\item Third level numbered list. Enumerate creates numbered lists, itemize creates bulleted lists and
description creates unnumbered lists.

\item Third level numbered list. Enumerate creates numbered lists, itemize creates bulleted lists and
description creates unnumbered lists.
\end{enumerate}

\item Second level numbered list. Enumerate creates numbered lists, itemize creates bulleted lists and
description creates unnumbered lists.

\item Second level numbered list. Enumerate creates numbered lists, itemize creates bulleted lists and
description creates unnumbered lists.
\end{enumerate}

\item Enumerate creates numbered lists, itemize creates bulleted lists and
description creates unnumbered lists.

\item Numbered lists continue.
\end{enumerate}
Lists in \LaTeX{} can be of three types: enumerate, itemize and description.
In each environment, a new entry is added via the \verb+\item+ command.
\begin{itemize}
\item First level bulleted list. This is the 1st item

\item First level bulleted list. Itemize creates bulleted lists and description creates unnumbered lists.
\begin{itemize}
\item Second level dashed list. Itemize creates bulleted lists and description creates unnumbered lists.

\item Second level dashed list. Itemize creates bulleted lists and description creates unnumbered lists.

\item Second level dashed list. Itemize creates bulleted lists and description creates unnumbered lists.
\end{itemize}

\item First level bulleted list. Itemize creates bulleted lists and description creates unnumbered lists.

\item First level bulleted list. Bullet lists continue.
\end{itemize}

\noindent
Example for unnumbered list items:

\begin{unlist}
\item Sample unnumberd list text. Sample unnumberd list text. Sample unnumberd list text. Sample unnumberd list text. Sample unnumberd list text.

\item Sample unnumberd list text. Sample unnumberd list text. Sample unnumberd list text.

\item sample unnumberd list text. Sample unnumberd list text. Sample unnumberd list text. Sample unnumberd list text. Sample unnumberd list text. Sample unnumberd list text. Sample unnumberd list text.
\end{unlist}

\section{Examples for theorem-like environments}\label{sec10}

For theorem-like environments, we require the \verb+amsthm+ package. There are three types of predefined theorem styles - \verb+thmstyleone+, \verb+thmstyletwo+ and \verb+thmstylethree+   (check your journal's instructions page in case a specific style is required).

\medskip
\noindent\begin{tabular}{|l|p{13pc}|}
\hline
\verb+thmstyleone+ & Numbered, theorem head in bold font and theorem text in italic style \\\hline
\verb+thmstyletwo+ & Numbered, theorem head in roman font and theorem text in italic style \\\hline
\verb+thmstylethree+ & Numbered, theorem head in bold font and theorem text in roman style \\\hline
\end{tabular}


\begin{theorem}[Theorem subhead]\label{thm1}
Example theorem text. Example theorem text. Example theorem text. Example theorem text. Example theorem text.
Example theorem text. Example theorem text. Example theorem text. Example theorem text. Example theorem text.
Example theorem text.
\end{theorem}

Quisque ullamcorper placerat ipsum. Cras nibh. Morbi vel justo vitae lacus tincidunt ultrices. Lorem ipsum dolor sit
amet, consectetuer adipiscing elit. In hac habitasse platea dictumst. Integer tempus convallis augue.

\begin{proposition}
Example proposition text. Example proposition text. Example proposition text. Example proposition text. Example proposition text.
Example proposition text. Example proposition text. Example proposition text. Example proposition text. Example proposition text.
\end{proposition}

Nulla malesuada porttitor diam. Donec felis erat, congue non, volutpat at, tincidunt tristique, libero. Vivamus
viverra fermentum felis. Donec nonummy pellentesque ante.

\begin{example}
Phasellus adipiscing semper elit. Proin fermentum massa
ac quam. Sed diam turpis, molestie vitae, placerat a, molestie nec, leo. Maecenas lacinia. Nam ipsum ligula, eleifend
at, accumsan nec, suscipit a, ipsum. Morbi blandit ligula feugiat magna. Nunc eleifend consequat lorem.
\end{example}

Nulla malesuada porttitor diam. Donec felis erat, congue non, volutpat at, tincidunt tristique, libero. Vivamus
viverra fermentum felis. Donec nonummy pellentesque ante.

\begin{remark}
Phasellus adipiscing semper elit. Proin fermentum massa
ac quam. Sed diam turpis, molestie vitae, placerat a, molestie nec, leo. Maecenas lacinia. Nam ipsum ligula, eleifend
at, accumsan nec, suscipit a, ipsum. Morbi blandit ligula feugiat magna. Nunc eleifend consequat lorem.
\end{remark}

Quisque ullamcorper placerat ipsum. Cras nibh. Morbi vel justo vitae lacus tincidunt ultrices. Lorem ipsum dolor sit
amet, consectetuer adipiscing elit. In hac habitasse platea dictumst.

\begin{definition}[Definition sub head]
Example definition text. Example definition text. Example definition text. Example definition text. Example definition text. Example definition text. Example definition text. Example definition text.
\end{definition}

Apart from the above styles, we have the \verb+\begin{proof}+ \verb+...+ \verb+\end{proof}+ environment - with the proof head in italic style and the body text in roman font with an open square at the end of each proof environment.

\begin{proof}Example for proof text. Example for proof text. Example for proof text. Example for proof text. Example for proof text. Example for proof text. Example for proof text. Example for proof text. Example for proof text. Example for proof text.
\end{proof}

Nam dui ligula, fringilla a, euismod sodales, sollicitudin vel, wisi. Morbi auctor lorem non justo. Nam lacus libero,
pretium at, lobortis vitae, ultricies et, tellus. Donec aliquet, tortor sed accumsan bibendum, erat ligula aliquet magna,
vitae ornare odio metus a mi.

\begin{proof}[Proof of Theorem~{\upshape\ref{thm1}}]
Example for proof text. Example for proof text. Example for proof text. Example for proof text. Example for proof text. Example for proof text. Example for proof text. Example for proof text. Example for proof text. Example for proof text.
\end{proof}

\noindent
For a quote environment, one has to use\newline \verb+\begin{quote}...\end{quote}+
\begin{quote}
Quoted text example. Aliquam porttitor quam a lacus. Praesent vel arcu ut tortor cursus volutpat. In vitae pede quis diam bibendum placerat. Fusce elementum
convallis neque. Sed dolor orci, scelerisque ac, dapibus nec, ultricies ut, mi. Duis nec dui quis leo sagittis commodo.
\end{quote}
Donec congue. Maecenas urna mi, suscipit in, placerat ut, vestibulum ut, massa. Fusce ultrices nulla et nisl (refer Figure~\ref{fig3}). Pellentesque habitant morbi tristique senectus et netus et malesuada fames ac turpis egestas. Etiam ligula arcu,
elementum a, venenatis quis, sollicitudin sed, metus. Donec nunc pede, tincidunt in, venenatis vitae, faucibus vel (refer Table~\ref{tab3}).

\section{Conclusion}

Some Conclusions here.

%%%%%%%%%%%%%%

\begin{appendices}

\section{Section title of first appendix}\label{sec11}

Nam dui ligula, fringilla a, euismod sodales, sollicitudin vel, wisi. Morbi auctor lorem non justo. Nam lacus libero,
pretium at, lobortis vitae, ultricies et, tellus. Donec aliquet, tortor sed accumsan bibendum, erat ligula aliquet magna,
vitae ornare odio metus a mi. Morbi ac orci et nisl hendrerit mollis. Suspendisse ut massa. Cras nec ante. Pellentesque
a nulla. Cum sociis natoque penatibus et magnis dis parturient montes, nascetur ridiculus mus. Aliquam tincidunt
urna. Nulla ullamcorper vestibulum turpis. Pellentesque cursus luctus mauris.

\subsection{Subsection title of first appendix}\label{subsec4}

Nam dui ligula, fringilla a, euismod sodales, sollicitudin vel, wisi. Morbi auctor lorem non justo. Nam lacus libero,
pretium at, lobortis vitae, ultricies et, tellus. Donec aliquet, tortor sed accumsan bibendum, erat ligula aliquet magna,
vitae ornare odio metus a mi. Morbi ac orci et nisl hendrerit mollis. Suspendisse ut massa. Cras nec ante. Pellentesque
a nulla. Cum sociis natoque penatibus et magnis dis parturient montes, nascetur ridiculus mus. Aliquam tincidunt
urna. Nulla ullamcorper vestibulum turpis. Pellentesque cursus luctus mauris.

\subsubsection{Subsubsection title of first appendix}\label{subsubsec3}

Example for an unnumbered figure:

\begin{figure}[!h]
\centering
{\color{black!20}\rule{85pt}{92pt}}
\end{figure}

Fusce mauris. Vestibulum luctus nibh at lectus. Sed bibendum, nulla a faucibus semper, leo velit ultricies tellus, ac
venenatis arcu wisi vel nisl. Vestibulum diam. Aliquam pellentesque, augue quis sagittis posuere, turpis lacus congue
quam, in hendrerit risus eros eget felis.

\section{Section title of second appendix}\label{sec12}%

Fusce mauris. Vestibulum luctus nibh at lectus. Sed bibendum, nulla a faucibus semper, leo velit ultricies tellus, ac
venenatis arcu wisi vel nisl. Vestibulum diam. Aliquam pellentesque, augue quis sagittis posuere, turpis lacus congue
quam, in hendrerit risus eros eget felis. Maecenas eget erat in sapien mattis porttitor. Vestibulum porttitor. Nulla
facilisi. Sed a turpis eu lacus commodo facilisis. Morbi fringilla, wisi in dignissim interdum, justo lectus sagittis dui, et
vehicula libero dui cursus dui. Mauris tempor ligula sed lacus. Duis cursus enim ut augue. Cras ac magna. Cras nulla.

\begin{figure}[b]
\centering
{\color{black!20}\rule{217pt}{120pt}}
\caption{This is an example for appendix figure\label{fig4}}
\end{figure}

\begin{table}[t]%
\begin{center}
\begin{minipage}{.52\columnwidth}
\caption{This is an example of Appendix table showing food requirements of army, navy and airforce\label{tab4}}%
\begin{tabular}{@{}lcc@{}}%
\toprule
col1 head & col2 head & col3 head \\
\midrule
col1 text & col2 text & col3 text \\
col1 text & col2 text & col3 text \\
col1 text & col2 text & col3 text \\
\botrule
\end{tabular}
\end{minipage}
\end{center}
\end{table}

\subsection{Subsection title of second appendix}\label{subsec5}

Sed commodo posuere pede. Mauris ut est. Ut quis purus. Sed ac odio. Sed vehicula hendrerit sem. Duis non odio.
Morbi ut dui. Sed accumsan risus eget odio. In hac habitasse platea dictumst. Pellentesque non elit. Fusce sed justo
eu urna porta tincidunt. Mauris felis odio, sollicitudin sed, volutpat a, ornare ac, erat. Morbi quis dolor. Donec
pellentesque, erat ac sagittis semper, nunc dui lobortis purus, quis congue purus metus ultricies tellus. Proin et quam.
Class aptent taciti sociosqu ad litora torquent per conubia nostra, per inceptos hymenaeos. Praesent sapien turpis,
fermentum vel, eleifend faucibus, vehicula eu, lacus.

Sed commodo posuere pede. Mauris ut est. Ut quis purus. Sed ac odio. Sed vehicula hendrerit sem. Duis non odio.
Morbi ut dui. Sed accumsan risus eget odio. In hac habitasse platea dictumst. Pellentesque non elit. Fusce sed justo
eu urna porta tincidunt. Mauris felis odio, sollicitudin sed, volutpat a, ornare ac, erat. Morbi quis dolor. Donec
pellentesque, erat ac sagittis semper, nunc dui lobortis purus, quis congue purus metus ultricies tellus. Proin et quam.
Class aptent taciti sociosqu ad litora torquent per conubia nostra, per inceptos hymenaeos. Praesent sapien turpis,
fermentum vel, eleifend faucibus, vehicula eu, lacus.

\subsubsection{Subsubsection title of second appendix}\label{subsubsec4}

Lorem ipsum dolor sit amet, consectetuer adipiscing elit. Ut purus elit, vestibulum ut, placerat ac, adipiscing vitae,
felis. Curabitur dictum gravida mauris. Nam arcu libero, nonummy eget, consectetuer id, vulputate a, magna. Donec
vehicula augue eu neque.


Example for an equation inside the appendix:
\begin{align}
 p &= \frac{\gamma^{2} - (n_{C} -1)H}{(n_{C} - 1) + H - 2\gamma}, \label{1eq:hybobo:pfromgH} \\
 \theta &= \frac{(\gamma - H)^{2}(\gamma - n_{C} -1)^{2}}{(n_{C} - 1 + H - 2\gamma)^{2}} \label{2eq:hybobo:tfromgH}\; .
\end{align}

\section{Example of another appendix section}\label{sec13}%

Nam dui ligula, fringilla a, euismod sodales, sollicitudin vel, wisi. Morbi auctor lorem non justo. Nam lacus libero,
pretium at, lobortis vitae, ultricies et, tellus. Donec aliquet, tortor sed accumsan bibendum, erat ligula aliquet magna,
vitae ornare odio metus a mi. Morbi ac orci et nisl hendrerit mollis. Suspendisse ut massa. Cras nec ante. Pellentesque
a nulla. Cum sociis natoque penatibus et magnis dis parturient montes, nascetur ridiculus mus. Aliquam tincidunt
urna. Nulla ullamcorper vestibulum turpis. Pellentesque cursus luctus mauris
\begin{equation}
\mathcal{L} = i \bar{\psi} \gamma^\mu D_\mu \psi
    - \frac{1}{4} F_{\mu\nu}^a F^{a\mu\nu} - m \bar{\psi} \psi.
\label{eq26}
\end{equation}

Nulla malesuada porttitor diam. Donec felis erat, congue non, volutpat at, tincidunt tristique, libero. Vivamus
viverra fermentum felis. Donec nonummy pellentesque ante. Phasellus adipiscing semper elit. Proin fermentum massa
ac quam. Sed diam turpis, molestie vitae, placerat a, molestie nec, leo. Maecenas lacinia. Nam ipsum ligula, eleifend
at, accumsan nec, suscipit a, ipsum. Morbi blandit ligula feugiat magna. Nunc eleifend consequat lorem. Sed lacinia
nulla vitae enim. Pellentesque tincidunt purus vel magna. Integer non enim. Praesent euismod nunc eu purus. Donec
bibendum quam in tellus. Nullam cursus pulvinar lectus. Donec et mi. Nam vulputate metus eu enim. Vestibulum
pellentesque felis eu massa.

Nulla malesuada porttitor diam. Donec felis erat, congue non, volutpat at, tincidunt tristique, libero. Vivamus
viverra fermentum felis. Donec nonummy pellentesque ante. Phasellus adipiscing semper elit. Proin fermentum massa
ac quam. Sed diam turpis, molestie vitae, placerat a, molestie nec, leo. Maecenas lacinia. Nam ipsum ligula, eleifend
at, accumsan nec, suscipit a, ipsum. Morbi blandit ligula feugiat magna. Nunc eleifend consequat lorem. Sed lacinia
nulla vitae enim. Pellentesque tincidunt purus vel magna. Integer non enim. Praesent euismod nunc eu purus. Donec
bibendum quam in tellus. Nullam cursus pulvinar lectus. Donec et mi. Nam vulputate metus eu enim. Vestibulum
pellentesque felis eu massa.

Nulla malesuada porttitor diam. Donec felis erat, congue non, volutpat at, tincidunt tristique, libero. Vivamus
viverra fermentum felis. Donec nonummy pellentesque ante. Phasellus adipiscing semper elit. Proin fermentum massa
ac quam. Sed diam turpis, molestie vitae, placerat a, molestie nec, leo. Maecenas lacinia. Nam ipsum ligula, eleifend
at, accumsan nec, suscipit a, ipsum. Morbi blandit ligula feugiat magna. Nunc eleifend consequat lorem. Sed lacinia
nulla vitae enim. Pellentesque tincidunt purus vel magna. Integer non enim. Praesent euismod nunc eu purus. Donec
bibendum quam in tellus. Nullam cursus pulvinar lectus. Donec et mi. Nam vulputate metus eu enim. Vestibulum
pellentesque felis eu massa.

Nulla malesuada porttitor diam. Donec felis erat, congue non, volutpat at, tincidunt tristique, libero. Vivamus
viverra fermentum felis. Donec nonummy pellentesque ante. Phasellus adipiscing semper elit. Proin fermentum massa
ac quam. Sed diam turpis, molestie vitae, placerat a, molestie nec, leo. Maecenas lacinia. Nam ipsum ligula, eleifend
at, accumsan nec, suscipit a, ipsum. Morbi blandit ligula feugiat magna. Nunc eleifend consequat lorem. Sed lacinia
nulla vitae enim. Pellentesque tincidunt purus vel magna. Integer non enim. Praesent euismod nunc eu purus. Donec
bibendum quam in tellus. Nullam cursus pulvinar lectus. Donec et mi. Nam vulputate metus eu enim. Vestibulum
pellentesque felis eu massa. Donec
bibendum quam in tellus. Nullam cursus pulvinar lectus. Donec et mi. Nam vulputate metus eu enim. Vestibulum
pellentesque felis eu massa.

%% Example for unnumbered table inside appendix
\begin{table}
\begin{center}
\begin{minipage}{.52\columnwidth}
\caption{}{%
\begin{tabular}{lcc}%
\toprule
col1 head & col2 head & col3 head \\
\midrule
col1 text & col2 text & col3 text \\
col1 text & col2 text & col3 text \\
col1 text & col2 text & col3 text \\
\botrule
\end{tabular}}{}
\end{minipage}
\end{center}
\end{table}

\end{appendices}

\section{Competing interests}
No competing interest is declared.

\section{Author contributions statement}

Must include all authors, identified by initials, for example:
S.R. and D.A. conceived the experiment(s),  S.R. conducted the experiment(s), S.R. and D.A. analysed the results.  S.R. and D.A. wrote and reviewed the manuscript.

\section{Acknowledgments}
The authors thank the anonymous reviewers for their valuable suggestions. This work is supported in part by funds from the National Science Foundation (NSF: \# 1636933 and \# 1920920).


%\bibliographystyle{plain}
%\bibliography{reference}

\begin{thebibliography}{10}

\bibitem{bahdanau2014neural}
Dzmitry Bahdanau, Kyunghyun Cho, and Yoshua Bengio.
\newblock Neural machine translation by jointly learning to align and
  translate.
\newblock {\em arXiv preprint arXiv:1409.0473}, 2014.

\bibitem{horvath2018dna}
Steve Horvath and Kenneth Raj.
\newblock Dna methylation-based biomarkers and the epigenetic clock theory of
  ageing.
\newblock {\em Nature Reviews Genetics}, 19(6):371, 2018.

\bibitem{imboden2018cardiorespiratory}
Mary~T Imboden, Matthew~P Harber, Mitchell~H Whaley, W~Holmes Finch, Derron~L
  Bishop, and Leonard~A Kaminsky.
\newblock Cardiorespiratory fitness and mortality in healthy men and women.
\newblock {\em Journal of the American College of Cardiology},
  72(19):2283--2292, 2018.

\bibitem{ji20123d}
Shuiwang Ji, Wei Xu, Ming Yang, and Kai Yu.
\newblock 3d convolutional neural networks for human action recognition.
\newblock {\em IEEE Transactions on Pattern Analysis and Machine Intelligence},
  35(1):221--231, 2012.

\bibitem{krizhevsky2012imagenet}
Alex Krizhevsky, Ilya Sutskever, and Geoffrey~E Hinton.
\newblock Imagenet classification with deep convolutional neural networks.
\newblock In {\em Advances in Neural Information Processing Systems}, pages
  1097--1105, 2012.

\bibitem{lecun2015deep}
Yann LeCun, Yoshua Bengio, and Geoffrey Hinton.
\newblock Deep learning.
\newblock {\em Nature}, 521(7553):436, 2015.

\bibitem{motiian2017unified}
Saeid Motiian, Marco Piccirilli, Donald~A Adjeroh, and Gianfranco Doretto.
\newblock Unified deep supervised domain adaptation and generalization.
\newblock In {\em Proceedings of the IEEE International Conference on Computer
  Vision}, pages 5715--5725, 2017.

\bibitem{murphy2012machine}
Kevin~P Murphy.
\newblock {\em Machine learning: A probabilistic perspective}.
\newblock MIT press, 2012.

\bibitem{american2013acsm}
American~College of~Sports~Medicine et~al.
\newblock {\em ACSM's guidelines for exercise testing and prescription}.
\newblock Lippincott Williams \& Wilkins, 2013.

\bibitem{pyrkov2018quantitative}
Timothy~V Pyrkov, Evgeny Getmantsev, Boris Zhurov, Konstantin Avchaciov,
  Mikhail Pyatnitskiy, Leonid Menshikov, Kristina Khodova, Andrei~V Gudkov, and
  Peter~O Fedichev.
\newblock Quantitative characterization of biological age and frailty based on
  locomotor activity records.
\newblock {\em Aging (Albany NY)}, 10(10):2973, 2018.

\bibitem{rahman2019centroidb}
Syed~Ashiqur Rahman and Donald Adjeroh.
\newblock Centroid of age neighborhoods: A generalized approach to estimate
  biological age.
\newblock In {\em 2019 IEEE EMBS International Conference on Biomedical \&
  Health Informatics (BHI)}, pages 1--4. IEEE, 2019.

\bibitem{ravi2016deep}
Daniele Rav{\`\i}, Charence Wong, Fani Deligianni, Melissa Berthelot, Javier
  Andreu-Perez, Benny Lo, and Guang-Zhong Yang.
\newblock Deep learning for health informatics.
\newblock {\em IEEE {J}ournal of {B}iomedical and {H}ealth {I}nformatics},
  21(1):4--21, 2016.

\bibitem{wang2018face}
Zongwei Wang, Xu~Tang, Weixin Luo, and Shenghua Gao.
\newblock Face aging with identity-preserved conditional generative adversarial
  networks.
\newblock In {\em Proceedings of the IEEE Conference on Computer Vision and
  Pattern Recognition}, pages 7939--7947, 2018.

\bibitem{zhang2018fine}
Ke~Zhang, Na~Liu, Xingfang Yuan, Xinyao Guo, Ce~Gao, and Zhenbing Zhao.
\newblock Fine-grained age estimation in the wild with attention {LSTM}
  networks.
\newblock {\em arXiv preprint arXiv:1805.10445}, 2018.

\end{thebibliography}


%USE THE BELOW OPTIONS IN CASE YOU NEED AUTHOR YEAR FORMAT.
%\bibliographystyle{abbrvnat}
%\bibliography{reference}



%% sample for biography with author's image
\begin{biography}{{\color{black!20}\rule{77pt}{77pt}}}{\author{Author Name.} This is sample author biography text. The values provided in the optional argument are meant for sample purposes. There is no need to include the width and height of an image in the optional argument for live articles. This is sample author biography text this is sample author biography text this is sample author biography text this is sample author biography text this is sample author biography text this is sample author biography text this is sample author biography text this is sample author biography text.}
\end{biography}

%% sample for biography without author's image
\begin{biography}{}{\author{Author Name.} This is sample author biography text this is sample author biography text this is sample author biography text this is sample author biography text this is sample author biography text this is sample author biography text this is sample author biography text this is sample author biography text.}
\end{biography}

\end{document}
